\documentclass[letterpaper]{article}
\usepackage{calc,amsmath,amssymb,amsfonts}
\usepackage[T1]{fontenc}
\usepackage[english,spanish]{babel}
\usepackage{xcolor,fancyhdr}
\usepackage[top=0.9839in,bottom=0.5in,hmargin=1.1811in,nohead,includefoot,foot=0.4839in,footskip=0.92840004in]{geometry}
\usepackage{array,supertabular,hhline,enumitem,hyperref}
\hypersetup{colorlinks=true,allcolors=blue,pdfauthor=Kevin Anthony Rogel Hern[FFFD?]ndez}
% Outline numbering
\setcounter{secnumdepth}{0}
% Text styles
\newcommand\textstyleInternetlink[1]{\textcolor[HTML]{0563C1}{#1}}
\makeatletter
\newcommand\arraybslash{\let\\\@arraycr}
\makeatother
% Pages
\fancypagestyle{Standard}{\fancyhf{}
  \fancyhead[L]{}
  \fancyfoot[C]{\thepage{}}
  \renewcommand\headrulewidth{0pt}
  \renewcommand\footrulewidth{0pt}
  \renewcommand\thepage{\arabic{page}}
}
\pagestyle{Standard}
\setlength\tabcolsep{1mm}
\renewcommand\arraystretch{1.3}
\author{Kevin Anthony Rogel Hern[FFFD?]ndez}
\date{2023-10-08}
\begin{document}
\clearpage
\pagestyle{Standard}
\setcounter{page}{1}
\section{Marco teórico: Herramientas Automatizadas de Prueba de Software}
\subsection{Generalidades}
Las herramientas automatizadas de prueba de software son programas o aplicaciones diseñados para simplificar y
automatizar el proceso de prueba de software, lo que resulta fundamental en el ciclo de vida del desarrollo de software
al facilitar la detección eficiente de problemas, errores y defectos en el software. A continuación, se presentan
algunas consideraciones generales acerca de estas herramientas:

\begin{itemize}[series=listWWNumx,]
\item \textbf{Automatización de pruebas:} El propósito principal de estas herramientas es automatizar la ejecución de
pruebas en un software. Esto se logra mediante la creación de secuencias de comandos o casos de prueba que se ejecutan
de manera programada, sin intervención manual, para evaluar el comportamiento del software.
\item \textbf{Tipos de pruebas:} Las herramientas de prueba automatizada se aplican a diversos tipos de pruebas, que
abarcan desde pruebas unitarias y de integración hasta pruebas funcionales, de rendimiento y de seguridad, entre otras.
\item \textbf{Ventajas de la automatización:} Entre las ventajas más destacadas de la automatización de pruebas se
incluyen la reducción de errores humanos, la ejecución rápida y repetible de pruebas, una cobertura exhaustiva de casos
de prueba y la capacidad de llevar a cabo pruebas de regresión de manera eficaz.
\item \textbf{Lenguajes de scripting:} A menudo, estas herramientas requieren que los profesionales de pruebas de
software desarrollen scripts o secuencias de comandos para definir las pruebas. Los lenguajes de scripting comunes
incluyen Selenium WebDriver, Appium, JUnit, TestNG, Python y otros.
\item \textbf{Integración continua:} Estas herramientas se integran sin dificultad en los flujos de trabajo de
integración continua (CI) y entrega continua (CD), lo que permite la detección temprana de problemas y una entrega de
software más rápida y segura.
\item \textbf{Herramientas populares:} Existen muchas herramientas de prueba automatizada disponibles, cada una con sus
propias características y ventajas. Algunas de las herramientas más conocidas incluyen Selenium, JUnit, TestNG, Appium,
Cucumber, Jenkins y muchas más.
\item \textbf{Requisitos de habilidades:} Para utilizar estas herramientas con eficacia, los profesionales de pruebas de
software deben contar con conocimientos en programación, comprensión de las metodologías de desarrollo de software y
experiencia en la creación de casos de prueba.
\item \textbf{Evaluación y selección:} Antes de elegir una herramienta de prueba automatizada, es esencial evaluar las
necesidades de prueba específicas de tu proyecto y seleccionar la herramienta que mejor se adapte a esos requisitos.
\item \textbf{Costos y licencias:} Las herramientas de prueba automatizada pueden variar en cuanto a costos, desde
opciones de código abierto gratuitas hasta soluciones comerciales más costosas. Es importante considerar los gastos de
licencia, soporte y mantenimiento al seleccionar una herramienta.
\end{itemize}
\section{Herramientas más utilizadas}
\subsection{Selenium WebDriver.}
Es una herramienta ampliamente reconocida y muy utilizada en la automatización de pruebas de software, especialmente
cuando se trata de evaluar aplicaciones web. En el ámbito de la automatización de pruebas, Selenium se considera una
colección de utilidades que ofrecen un conjunto completo de capacidades para automatizar la comunicación entre un
navegador web y una aplicación web.

\subsubsection{Características.}
\begin{enumerate}[series=listWWNumix,label=\arabic*.,ref=\arabic*]
\item \textbf{Compatibilidad con múltiples navegadores:} Es compatible con una amplia variedad de navegadores web,
incluidos Google Chrome, Mozilla Firefox, Microsoft Edge, Safari e Internet Explorer. Esto permite realizar pruebas en
diferentes navegadores para garantizar la compatibilidad cruzada de una aplicación web.
\item \textbf{Soporte multiplataforma:} Es compatible con varios sistemas operativos, como Windows, macOS y Linux. Esto
facilita la ejecución de pruebas en diferentes entornos.
\item \textbf{Lenguajes de programación:} Selenium WebDriver admite múltiples lenguajes de programación populares, como
Java, Python, C\#, Ruby, JavaScript y otros. Esto permite a los desarrolladores y probadores utilizar el lenguaje con
el que se sientan más cómodos para escribir scripts de automatización de pruebas.
\item \textbf{Grabación y reproducción (Selenium IDE):} Aunque no es la característica principal de Selenium, Selenium
IDE proporciona una extensión para navegadores que permite grabar y reproducir acciones en una aplicación web. Es una
forma fácil de comenzar con la automatización de pruebas, especialmente para usuarios no técnicos.
\item \textbf{Amplia comunidad y recursos:} Cuenta con una comunidad activa de usuarios y una gran cantidad de recursos
disponibles en línea, como documentación, tutoriales y foros de soporte. Esto facilita la obtención de ayuda y la
resolución de problemas.
\item \textbf{Flexibilidad:} Es altamente flexible y permite una amplia gama de interacciones con elementos web, como
hacer clic, llenar formularios, navegar por páginas, validar contenido y más. Esto lo hace adecuado para pruebas
funcionales y de regresión.
\item \textbf{Integración con herramientas de gestión de pruebas y CI/CD:} Se integra fácilmente con herramientas de
gestión de pruebas como TestRail y con plataformas de integración continua (CI) como Jenkins. Esto permite automatizar
y orquestar el proceso de ejecución de pruebas dentro del ciclo de entrega de software.
\item \textbf{Pruebas de carga y rendimiento:} Aunque Selenium está diseñado principalmente para pruebas funcionales, se
pueden utilizar complementos y herramientas adicionales para realizar pruebas de carga y rendimiento en aplicaciones
web.
\item \textbf{Ecosistema en constante evolución:} Sigue mejorando y adaptándose a las cambiantes tecnologías web. Los
desarrolladores de Selenium continúan desarrollando nuevas versiones y características para mantener la herramienta
relevante.
\item \textbf{Pruebas de aplicaciones móviles:} Además de las pruebas web, ofrece capacidades para automatizar pruebas
en aplicaciones móviles a través de frameworks como Appium.
\end{enumerate}
\subsubsection{Contenido de Selenium suits.}
\begin{itemize}[series=listWWNumxii,]
\item \textbf{Suite de Herramientas Selenium:} Selenium no es una única herramienta, sino una colección de herramientas
relacionadas que se utilizan en conjunto para la automatización de pruebas web. Las herramientas principales en la
suite de Selenium incluyen:
\item \textbf{\textit{Selenium WebDriver: }}Esta es la parte central de Selenium y se utiliza para automatizar la
interacción con navegadores web. Permite escribir scripts en varios lenguajes de programación (como Java, Python, C\#,
etc.) para controlar un navegador web y realizar acciones en una aplicación web.
\item \textbf{\textit{Selenium IDE: }}Es una extensión del navegador que permite la grabación y reproducción de acciones
en un navegador web. Es una forma fácil de comenzar con la automatización de pruebas, pero es menos poderosa que
Selenium WebDriver.
\item \textbf{\textit{Selenium Grid: }}Es una herramienta que permite la ejecución de pruebas en paralelo en múltiples
navegadores y sistemas operativos. Esto es útil para probar la compatibilidad multiplataforma de una aplicación web.
\item \textbf{Automatización de Pruebas Web:} Se utiliza principalmente para la automatización de pruebas en
aplicaciones web. Permite a los testers y desarrolladores escribir scripts que interactúan con una aplicación web de la
misma manera que lo haría un usuario real. Esto incluye acciones como hacer clic en enlaces, llenar formularios,
navegar por páginas y verificar el contenido de la página.
\item \textbf{Lenguajes de Programación:} Selenium WebDriver es compatible con varios lenguajes de programación
populares, lo que facilita su uso por parte de diferentes equipos de desarrollo y prueba. Algunos de los lenguajes
comunes son Java, Python, C\#, Ruby y JavaScript.
\item \textbf{Navegadores Compatibles}: Es compatible con una variedad de navegadores web populares, como Chrome,
Firefox, Safari, Edge e Internet Explorer. Esto permite realizar pruebas de compatibilidad cruzada para garantizar que
una aplicación web funcione correctamente en diferentes navegadores.
\item \textbf{Pruebas Funcionales y de Regresión:} Se utiliza para realizar pruebas funcionales y de regresión en
aplicaciones web. Las pruebas funcionales aseguran que una aplicación funcione según lo previsto, mientras que las
pruebas de regresión verifican que las nuevas actualizaciones no hayan introducido errores en áreas previamente
funcionales.
\end{itemize}
\subsection{Appium}
Se trata de una herramienta de código abierto destinada a la automatización de pruebas de software en aplicaciones
móviles que funcionan en sistemas operativos como iOS, Android y Windows. Su desarrollo tuvo como objetivo proporcionar
una solución integral para automatizar pruebas en dispositivos móviles y aplicaciones, sin importar el sistema
operativo utilizado o el tipo de aplicación (ya sea nativa, híbrida o basada en web).

\subsubsection{Características }
\begin{enumerate}[series=listWWNumxi,label=\arabic*.,ref=\arabic*]
\item \textbf{Versatilidad en Plataformas:} Es conocido por su capacidad de automatización en múltiples plataformas, lo
que incluye iOS y Android. Esto significa que los equipos de desarrollo y pruebas pueden emplear una única suite de
pruebas para ambas plataformas.
\item \textbf{Compatibilidad con Diversos Tipos de Aplicaciones:} Es capaz de automatizar pruebas en una variedad de
tipos de aplicaciones móviles, incluyendo aplicaciones nativas (desarrolladas específicamente para la plataforma),
aplicaciones híbridas (que combinan componentes web y nativos), y aplicaciones web móviles.
\item \textbf{Flexibilidad de Lenguajes de Programación:} Es compatible con varios lenguajes de programación populares,
como Java, Python, C\#, Ruby, JavaScript, entre otros. Esto brinda a los equipos de desarrollo y pruebas la libertad de
elegir el lenguaje que prefieran para escribir sus scripts de automatización de pruebas.
\item \textbf{Basado en el Protocolo WebDriver:} Se basa en el protocolo WebDriver, lo que implica que las pruebas
pueden ser desarrolladas utilizando la API estándar de WebDriver. Esto resulta especialmente beneficioso para aquellos
que ya tienen experiencia previa con Selenium WebDriver, ya que les resultará familiar.
\item \textbf{Compatibilidad con Simuladores y Dispositivos Reales:} Tiene la capacidad de automatizar pruebas tanto en
simuladores/emuladores como en dispositivos móviles reales. Esto permite llevar a cabo pruebas en diversos escenarios y
configuraciones.
\item \textbf{Integración con Frameworks de Pruebas:} Se integra sin problemas con muchos frameworks de pruebas
ampliamente utilizados, simplificando así la implementación de pruebas de unidad, pruebas funcionales y pruebas de
regresión.
\item \textbf{Soporte Activo y Comunidad Vibrante:} Cuenta con una comunidad activa de usuarios y desarrolladores, lo
que significa que existe una gran cantidad de recursos, documentación y soporte en línea disponibles para ayudar a los
usuarios a aprender y utilizar la herramienta de manera efectiva.
\item \textbf{Pruebas en Aplicaciones Nativas y Navegadores Móviles:} Permite la automatización de pruebas tanto en
aplicaciones móviles nativas como en navegadores móviles. Esto es valioso para llevar a cabo pruebas de funcionalidad y
rendimiento en aplicaciones móviles.
\end{enumerate}
\subsection{Cucumber}
Cucumber es una herramienta de código abierto utilizada para realizar pruebas de comportamiento (Behavior-Driven
Development o BDD, por sus siglas en inglés). BDD es una metodología de desarrollo de software que pone énfasis en la
colaboración entre desarrolladores, probadores y expertos en el dominio para definir y comprender el comportamiento
deseado de un software, desde la perspectiva del usuario o el negocio.

Cucumber facilita la comunicación entre estos distintos roles al permitir que las especificaciones de comportamiento se
redacten en un lenguaje natural comprensible para todas las partes involucradas. Estas especificaciones se estructuran
en forma de escenarios, los cuales describen situaciones concretas que se anticipa que ocurran en el software.

\subsubsection{Características}
\begin{enumerate}[series=listWWNumxiii,label=\arabic*.,ref=\arabic*]
\item \textbf{Lenguaje Gherkin:} Utiliza un lenguaje específico llamado Gherkin para escribir especificaciones de
comportamiento de manera legible y comprensible. Gherkin utiliza palabras clave como
{\textquotedbl}Given{\textquotedbl} (Dado), {\textquotedbl}When{\textquotedbl} (Cuando),
{\textquotedbl}Then{\textquotedbl} (Entonces) y otras para describir el flujo de un escenario.
\item \textbf{Ejecución de Escenarios:} Puede ejecutar los escenarios escritos en Gherkin y verificar si el
comportamiento real del software coincide con las especificaciones definidas. Si los escenarios fallan, Cucumber genera
informes detallados que indican dónde y por qué fallaron.
\item \textbf{Soporte Multilingüe:} Es compatible con varios lenguajes de programación populares, como Java, Ruby,
Python, C\#, entre otros. Esto permite a los equipos de desarrollo y pruebas utilizar el lenguaje de programación con
el que están más familiarizados.
\item \textbf{Integración con Herramientas de Pruebas y CI/CD:} Se puede integrar con diversas herramientas de pruebas,
como Selenium para pruebas de interfaz de usuario, y con sistemas de integración continua (CI) y entrega continua (CD)
para automatizar la ejecución de pruebas como parte del flujo de trabajo de desarrollo.
\item \textbf{Reutilización de Pasos:} Los pasos comunes en los escenarios pueden reutilizarse en múltiples
especificaciones, lo que ayuda a mantener un conjunto coherente de pasos y facilita la escalabilidad de las pruebas.
\item \textbf{Colaboración:} Fomenta la colaboración entre desarrolladores, testers y otros interesados, ya que todos
pueden entender y revisar las especificaciones de comportamiento escritas en Gherkin.
\item \textbf{Pruebas de Aceptación y Validación:} Se utiliza principalmente para escribir pruebas de aceptación que
verifican si una aplicación cumple con los requisitos del usuario y el negocio. También se puede utilizar para validar
que nuevas características no introduzcan errores en las áreas existentes de la aplicación (pruebas de regresión).
\end{enumerate}
\subsection{Cuadro comparativo técnico de las herramientas}
\begin{flushleft}
\tablefirsthead{}
\tablehead{}
\tabletail{}
\tablelasttail{}
\centering
\begin{supertabular}{m{3.2cm}m{3.2cm}m{3.2cm}m{3.2cm}}
\hline
\centering{\bfseries Aspecto Técnico} &
\centering{\bfseries Selenium WebDtiver} &
\centering{\bfseries Appium} &
\centering\arraybslash{\bfseries Cucumber}\\\hline
\centering{\bfseries Tipo de Herramienta} &
\centering Automatización de pruebas web y aplicaciones &
\centering Automatización de pruebas móviles y aplicaciones &
\centering\arraybslash BDD para especificaciones y pruebas\\\hline
\centering{\bfseries Enfoque} &
\centering Pruebas de aplicaciones web en navegadores &
\centering Pruebas de aplicaciones móviles en iOS, Android y Windows &
\centering\arraybslash Pruebas de comportamiento y validación\\\hline
\centering{\bfseries Plataformas Soportadas} &
\centering Múltiples navegadores y sistemas operativos &
\centering iOS, Android, Windows &
\centering\arraybslash Múltiples lenguajes de programación y sistemas operativos\\\hline
\centering{\bfseries Soporte para Aplicaciones Nativas, Híbridas y Web} &
\centering Sí (mediante Selenium WebDriver) &
\centering Sí (Aplicaciones nativas, híbridas y web) &
\centering\arraybslash Enfoque en especificaciones, no directamente para tipos de aplicaciones\\\hline
\centering{\bfseries Compatibilidad con Browsers y Dispositivos} &
\centering Amplia compatibilidad con navegadores &
\centering Amplia compatibilidad con dispositivos móviles &
\centering\arraybslash No aplica, ya que no es una herramienta de automatización directa\\\hline
\centering{\bfseries Generación de Informes} &
\centering Integración con frameworks de informes (como TestNG, JUnit) &
\centering Generación de informes personalizados &
\centering\arraybslash Generación de informes integrados en el formato Gherkin\\\hline
\centering{\bfseries Integración con CI/CD} &
\centering Integración con herramientas de CI/CD como Jenkins &
\centering Integración con sistemas de CI/CD &
\centering\arraybslash Integración con herramientas de CI/CD para ejecución automatizada\\\hline
\centering{\bfseries Flexibilidad} &
\centering Muy flexible y versátil para pruebas web &
\centering Flexible y versátil para pruebas móviles &
\centering\arraybslash Flexible en la definición de comportamiento y pasos de prueba\\\hline
\end{supertabular}
\end{flushleft}

\bigskip

\subsection{Selección de herramienta para desarrollar un manual de instalación, ejemplos de aplicación y video de
funcionamiento de la herramienta.}
Se seleccionará la herramienta de Selenium WebDrive por ser la más conocida y contar con una comunidad activa y una
basta documentación de apoyo.

\subsubsection{Manual de instalación de Selenium. }
\begin{itemize}[series=listWWNumx,]
\item \textbf{Paso 1: Descargar Selenium.}

Descargar del sitio oficial de selenium webdrive \url{https://www.selenium.dev/downloads/}

\item \textbf{\textit{Paso 2: Descargar un entorno de desarrollo java junto al Java Development Kit (JDK)}}

Debemos descargar el JDK acompañado de un entorno de desarrollo IDE para java

\item \textbf{\textit{Paso 3: Configuramos un proyecto de java.}}

Creamos un proyecto de java en el IDE que hayamos instalado en nuestro computador.

\item \textbf{\textit{Paso 4: Agregar selenium webdriver a nuestro proyecto.}}

Haremos clic derecho en el proyecto y elige la opción {\textquotedbl}Properties{\textquotedbl} (Propiedades).

Navegaremos hasta la sección {\textquotedbl}Java Build Path{\textquotedbl} y seleccionaremos la pestaña
{\textquotedbl}Libraries{\textquotedbl}.

Seleccionaremos la opción {\textquotedbl}Add External JARs...{\textquotedbl} y busca los archivos JAR de Selenium que se
encuentran en la carpeta que descargamos en el paso 1, hay que tener en cuenta que la carpeta que descarguemos en el
paso 1 debemos de descomprimirla.

Asegúrate de que las bibliotecas de Selenium estén incluidas en la lista de bibliotecas del proyecto y, finalmente, haz
clic en {\textquotedbl}Apply and Close{\textquotedbl} y quedara lista para utilizarse.

\item \textbf{\textit{Paso 5: Descargar los controladores necesarios para el uso de Selenium.}}

Descargaremos los controladores específicos para cada navegador (seleccionar un navegador y descargar su controlador),
los cuales incluiremos en nuestro proyecto.

\textbf{\textit{Para Google Chrome:}}

Descarga el controlador de Chrome WebDriver desde https://sites.google.com/chromium.org/driver/.

\textbf{\textit{Para Mozilla Firefox:}}

Descarga el controlador de Gecko WebDriver desde \url{https://github.com/mozilla/geckodriver/releases}.

\item \textbf{\textit{Paso 6: Crear el primer script de prueba de Selenium.}}

Ahora que Selenium WebDriver está configurado en tu proyecto, puedes comenzar a escribir scripts de prueba en Java.

\item \textbf{\textit{Paso 7: Ejecutar script de Selenium}}

Compilaremos nuestro script de prueba, al ejecutarlo Selenium abrirá una ventana de navegador en la cual daremos por
hecho que ha funcionado. 
\end{itemize}
\subsubsection{Video tutorial de instalación y ejecución de Selenium.}
\url{https://youtu.be/OEiWZUgxWfQ?si=o4sfwgJp1fflMwXH}

\section{Marco teórico: Herramientas de documentación.}
Las herramientas de documentación son programas o software diseñados para facilitar la creación, organización, gestión y
compartición eficiente de documentos y contenido. Estas aplicaciones son valiosas en diversos entornos, desde empresas
y equipos de desarrollo hasta instituciones académicas y organizaciones sin fines de lucro. A continuación, se presenta
una descripción general de las principales funciones y categorías de herramientas de documentación:

\begin{enumerate}[series=listWWNumxiv,label=\arabic*.,ref=\arabic*]
\item \textbf{Creación de Documentos:} Estas herramientas posibilitan la elaboración de documentos en diferentes
formatos, como documentos de texto, informes técnicos, presentaciones, hojas de cálculo y diagramas. Ejemplos incluyen
Microsoft Word, Google Docs, LibreOffice Writer y aplicaciones de diagramación como Lucidchart.
\item \textbf{Gestión de Documentos:} Estas aplicaciones ayudan a organizar y administrar archivos, ofreciendo funciones
como almacenamiento en la nube, control de versiones, etiquetado y búsquedas avanzadas.
\foreignlanguage{english}{Ejemplos son SharePoint, Dropbox Business y Google Drive for Work.}
\item \textbf{Colaboración en Documentos:} Facilitan la colaboración en tiempo real en documentos compartidos,
permitiendo a múltiples usuarios editar y comentar simultáneamente en un mismo archivo. Ejemplos incluyen Google
Workspace (anteriormente G Suite), Microsoft Teams y Quip.
\item \textbf{Gestión del Ciclo de Vida de Documentos (DLM):} Estas herramientas ayudan a gestionar la creación,
revisión, aprobación, publicación y archivo de documentos a lo largo de su ciclo de vida. Son especialmente útiles en
industrias reguladas, como la farmacéutica y la construcción.
\item \textbf{Documentación Técnica:} Se enfocan en la creación y gestión de documentación técnica, manuales de usuario,
documentación de API y otros materiales técnicos. Ejemplos incluyen Confluence, DocuWare y MadCap Flare.
\end{enumerate}

\bigskip

\begin{enumerate}[resume*=listWWNumxiv]
\item \textbf{Documentación de Procesos:} Están diseñadas para crear diagramas de flujo, manuales de procedimientos y
documentación de procesos empresariales. Ejemplos son Microsoft Visio, Draw.io y Lucidchart.
\item \textbf{Herramientas de Captura de Pantalla:} Permiten tomar capturas de pantalla y añadir anotaciones para
explicar conceptos. Ejemplos incluyen Snagit y Greenshot.
\item \textbf{Gestión de Documentos Empresariales (EDM):} Son sistemas más complejos utilizados para gestionar
documentos en organizaciones grandes, incluyendo funciones de flujo de trabajo, cumplimiento normativo y seguridad
avanzada. Ejemplos incluyen OpenText Documentum y M-Files.
\item \textbf{Documentación Colaborativa en Línea:} Plataformas de documentación en línea que facilitan la colaboración
en la creación y edición de documentos directamente en la web. Ejemplos son Notion, Quip y Zoho Docs.
\item \textbf{Herramientas de Wiki:} Permiten la creación y edición colaborativa de contenidos tipo wiki, lo que
simplifica la creación de bases de conocimiento y documentación compartida. Ejemplos incluyen MediaWiki y Confluence.
\item \textbf{Documentación de Código:} Herramientas específicas para documentar código fuente, que incluyen comentarios
en el código, generación de documentación técnica y creación de manuales de usuario para aplicaciones y software.
\item \textbf{Herramientas de Escritura Creativa:} Destinadas a autores y escritores, estas aplicaciones ofrecen
funciones específicas para la escritura de novelas, guiones, ensayos y otros tipos de contenido creativo. Ejemplos
incluyen Scrivener y Ulysses.
\end{enumerate}
\subsection{Sphinx}
Sphinx es una herramienta muy empleada en el mundo del desarrollo de software y en diversos campos técnicos. Se utiliza
extensamente en proyectos de código abierto, para documentar bibliotecas y marcos de trabajo, así como en la creación
de manuales de usuario. Es útil en cualquier contexto donde se necesite documentación técnica precisa y actualizada. Su
capacidad para generar automáticamente documentación a partir del código fuente hace que sea una herramienta invaluable
para mantener la consistencia y la precisión de la documentación en proyectos de desarrollo de software.

\subsubsection{Características y funcionalidades}
\begin{enumerate}[series=listWWNumxv,label=\arabic*.,ref=\arabic*]
\item \textbf{Generación de Documentación Automática:} Permite generar documentación técnica de forma automática a
partir de comentarios en el código fuente y archivos de marcado. Es especialmente útil para proyectos de software donde
se necesita mantener documentación actualizada de manera eficiente.
\item \textbf{Compatibilidad con Diversos Lenguajes de Programación:} Es compatible con varios lenguajes de
programación, incluyendo Python, C/C++, Java, Ruby y otros. Esto lo hace adecuado para proyectos que utilizan una
variedad de tecnologías.
\item \textbf{Soporte para Diversos Formatos de Salida:} Puede generar documentación en varios formatos, como HTML, PDF,
ePub y más. Esto permite que la documentación sea accesible en diferentes dispositivos y plataformas.
\item \textbf{Facilidad de Uso:} Utiliza un formato de marcado llamado reStructuredText, que es fácil de aprender y
escribir. Los comentarios en el código fuente se pueden enriquecer con directivas de Sphinx para generar documentación
enriquecida.
\item \textbf{Temas y Personalización:} Ofrece una variedad de temas y opciones de personalización para que los
desarrolladores puedan adaptar la apariencia de la documentación a sus necesidades y preferencias.
\item \textbf{Generación de Índices y Referencias Cruzadas:} Genera automáticamente índices, tablas de contenido y
referencias cruzadas dentro de la documentación, lo que facilita la navegación y búsqueda.
\item \textbf{Integración con Herramientas de Construcción:} Se integra bien con sistemas de construcción y herramientas
de automatización, como Makefile y sistemas de control de versiones como Git.
\item \textbf{Extensiones y Plugins:} Es altamente extensible y admite una variedad de extensiones y plugins
desarrollados por la comunidad. Estas extensiones pueden agregar funcionalidades adicionales a la generación de
documentación.
\item \textbf{Documentación Técnica Compleja:} Es adecuado para proyectos con documentación técnica compleja, incluyendo
documentación de APIs, diagramas de clases, ejemplos de código, y más
\end{enumerate}
\subsubsection{Estándares de documentación}
\begin{enumerate}[series=listWWNumxvi,label=\arabic*.,ref=\arabic*]
\item \textbf{Estandarización del Estilo de Documentación:} Permite definir y aplicar un estilo de documentación
coherente en un proyecto. Esto puede incluir directrices para la estructura de la documentación, el formato de texto,
la organización de secciones y la presentación visual.
\item \textbf{Estándares de Comentarios en el Código Fuente:} Se integra con comentarios en el código fuente, lo que
facilita la adhesión a estándares de documentación de código, como los comentarios docstring en Python (por ejemplo,
PEP 257).
\item \textbf{Estandarización de Nombres y Convenciones:} Los nombres de archivos y directorios, así como las etiquetas
y referencias cruzadas, pueden seguir convenciones específicas definidas en el proyecto para garantizar una
documentación coherente y fácil de seguir.
\item \textbf{Documentación de APIs y Bibliotecas:} Es ampliamente utilizado para documentar APIs y bibliotecas de
software siguiendo estándares y prácticas recomendadas. Esto incluye la generación de documentación de clases,
funciones y métodos, así como la descripción de parámetros y valores de retorno.
\item \textbf{Adherencia a Estándares de Marcado:} Utiliza reStructuredText como formato de marcado. Los proyectos
pueden aplicar estándares de marcado específicos, como el uso de directivas y roles personalizados, para enriquecer la
documentación según sea necesario.
\item \textbf{Estándares de Presentación:} La apariencia visual de la documentación generada por Sphinx se puede
personalizar utilizando temas y estilos específicos. Esto permite que la documentación siga las pautas de diseño y
presentación de un proyecto.
\item \textbf{Gestión de Versiones y Control de Cambios:} Puede integrarse con sistemas de control de versiones como
Git, lo que facilita la gestión de cambios y la colaboración en la documentación, lo que es importante para mantener
estándares a lo largo del tiempo.
\item \textbf{Cumplimiento de Estándares Externos:} Es compatible con la generación de documentación en múltiples
formatos, como HTML, PDF y LaTeX. Esto permite que la documentación cumpla con estándares específicos para la
publicación en línea o la impresión.
\end{enumerate}
\subsection{LateX}
LaTeX es un sistema de composición de texto ampliamente utilizado para crear documentos de alta calidad, especialmente
en entornos académicos, científicos y técnicos. Su enfoque principal es la presentación profesional de documentos,
incluyendo notación matemática compleja, fórmulas, gráficos y referencias bibliográficas.

\subsubsection{Características}
\begin{enumerate}[series=listWWNumxvii,label=\arabic*.,ref=\arabic*]
\item \textbf{Propósito Principal:} LaTeX se utiliza para crear documentos de alta calidad con un enfoque en la
presentación profesional y precisa del contenido. Es especialmente adecuado para documentos que incluyen notación
matemática compleja, fórmulas, gráficos, referencias bibliográficas y contenido técnico.
\item \textbf{Estructura de Marcado:} Utiliza un sistema de marcado estructurado, lo que significa que los documentos se
crean describiendo su estructura y contenido utilizando comandos y etiquetas en lugar de formato visual directo. Esto
permite un alto grado de control sobre el formato y estilo de los documentos.
\item \textbf{Notación Matemática:} Es ampliamente conocido por su capacidad para representar notación matemática
compleja de manera clara y profesional. Es una elección común para la creación de documentos matemáticos, artículos
científicos y tesis académicas.
\item \textbf{Documentos Largos y Complejos:} Es especialmente útil para la creación de documentos largos y complejos,
como tesis doctorales, libros técnicos, manuales técnicos y documentos de investigación.
\item \textbf{Bibliografía y Citas:} Ofrece herramientas integradas para gestionar citas y referencias bibliográficas.
El sistema BibTeX es ampliamente utilizado para crear y mantener listas de referencias bibliográficas.
\item \textbf{Generación de Índices y Tablas de Contenido:} Genera automáticamente índices, tablas de contenido y listas
de figuras y tablas, lo que facilita la navegación y búsqueda en documentos extensos.
\item \textbf{Personalización y Plantillas:} Los usuarios pueden personalizar el formato y el estilo de los documentos
mediante la definición de plantillas y estilos personalizados. Además, hay muchas plantillas predefinidas disponibles
para diferentes tipos de documentos.
\item \textbf{Portabilidad:} Los documentos son archivos de texto plano, lo que los hace altamente portátiles y
compatibles con una amplia variedad de sistemas operativos y software de edición LaTeX.
\item \textbf{Entorno de Desarrollo:} Los usuarios pueden escribir documentos LaTeX en un entorno de desarrollo
específico (como TeXShop, TeXworks o Overleaf) o en editores de texto simples con soporte para LaTeX.
\item \textbf{Salida en Diversos Formatos:} Además de la salida en formato PDF, LaTeX puede generar documentos en
formatos como DVI (Device Independent), PostScript y HTML, entre otros.
\end{enumerate}
\subsubsection{Estándares de documentación}
LaTeX, como sistema de composición de texto, no impone estándares específicos de documentación, pero es altamente
adaptable y permite a los usuarios aplicar sus propios estándares y convenciones de acuerdo con las necesidades de
documentación de su proyecto. Consideraciones:

\begin{enumerate}[series=listWWNumxviii,label=\arabic*.,ref=\arabic*]
\item \textbf{Estructura de Documentos:} Es común establecer una estructura de documentos consistente, que puede incluir
secciones como portada, resumen, índice, introducción, capítulos, conclusiones, bibliografía, entre otros. La forma en
que se organizan y nombran estas secciones puede seguir un estándar específico.
\item \textbf{Estilo de Texto:} Permite el control detallado del estilo de texto, como la fuente, el tamaño de fuente,
la alineación, el espaciado y los márgenes. Se pueden definir estándares de estilo de texto para garantizar la
uniformidad en todo el documento.
\item \textbf{Notación Matemática:} Si el documento incluye notación matemática o fórmulas, es importante seguir
estándares de notación matemática para garantizar la precisión y la comprensión. LaTeX es conocido por su capacidad
para representar notación matemática de manera profesional.
\item \textbf{Citas y Referencias:} Para la gestión de citas y referencias bibliográficas, se pueden seguir estándares
como el estilo APA, MLA, Chicago, etc. LaTeX ofrece herramientas como BibTeX para facilitar la gestión de
bibliografías.
\item \textbf{Numéricos y Unidades:} Cuando se presenten números y unidades, es importante seguir estándares de notación
numérica y unidades de medida, como el Sistema Internacional de Unidades (SI).
\item \textbf{Diagramas y Gráficos:} Si se incluyen diagramas, gráficos o figuras, es útil mantener un estándar de
formato y etiquetado para facilitar la comprensión.
\item \textbf{Estándares de Presentación:} La presentación visual del documento, incluyendo la elección de colores,
estilos de encabezado y pie de página, y otros elementos de diseño, puede seguir estándares específicos de la
organización o el proyecto.
\item \textbf{Consistencia de Términos:} Es importante ser consistente en la terminología utilizada a lo largo del
documento. Se pueden establecer estándares para la terminología y definiciones.
\item \textbf{Formato de Página:} Los estándares de formato de página, como el tamaño de página, los márgenes y la
orientación (vertical u horizontal), deben ser consistentes en todo el documento.
\item \textbf{Control de Versiones:} Si se utiliza un sistema de control de versiones como Git, es útil seguir prácticas
de control de versiones para documentación, como incluir marcas de tiempo o números de versión.
\end{enumerate}

\bigskip

\subsection{Cuadro comparativo técnico de las herramientas de documentación }
\begin{flushleft}
\tablefirsthead{}
\tablehead{}
\tabletail{}
\tablelasttail{}
\centering
\begin{supertabular}{m{4cm}m{4cm}m{4cm}}
\hline
\centering{\bfseries Aspecto} &
\centering{\bfseries LaTeX} &
\centering\arraybslash{\bfseries Sphinx}\\\hline
\centering{\bfseries Propósito Principal} &
\centering Composición de documentos de alta calidad, especialmente en ámbitos académicos y científicos. &
\centering\arraybslash Generación de documentación técnica a partir del código fuente.\\\hline
\centering{\bfseries Estructura de Marcado} &
\centering Uso de comandos y etiquetas de marcado LaTeX para estructurar documentos. &
\centering\arraybslash Uso de reStructuredText (reST), un formato de marcado estructurado, para describir la estructura
del documento.\\\hline
\centering{\bfseries Citas y Referencias} &
\centering Ofrece herramientas como BibTeX para gestionar citas y referencias bibliográficas. &
\centering\arraybslash Admite referencias cruzadas y genera automáticamente índices y tablas de contenido.\\\hline
\centering{\bfseries Diagramas y Gráficos} &
\centering Permite la inclusión de figuras y gráficos, pero requiere preparar los gráficos por separado. &
\centering\arraybslash Se enfoca en la documentación de texto y no incluye capacidades de gráficos directamente.\\\hline
\centering{\bfseries Integración de Código} &
\centering No está diseñado para incluir y formatear código fuente en la documentación. &
\centering\arraybslash Admite la inclusión y formateo de código fuente en la documentación.\\\hline
\centering{\bfseries Control de Versiones} &
\centering Los archivos LaTeX son archivos de texto plano y son compatibles con sistemas de control de versiones. &
\centering\arraybslash Puede integrarse con sistemas de control de versiones para gestionar cambios en la
documentación.\\\hline
\centering{\bfseries Salida en Varios Formatos} &
\centering Puede generar documentos en formatos como PDF, DVI, PostScript, etc. &
\centering\arraybslash Puede generar documentos en HTML, PDF, y otros formatos.\\\hline
\centering{\bfseries Nivel de Dificultad} &
\centering Puede tener una curva de aprendizaje más pronunciada debido a su marcado específico. &
\centering\arraybslash Más accesible y fácil de aprender para usuarios técnicos.\\\hline
\end{supertabular}
\end{flushleft}

\bigskip

\subsection{Seleccionar herramienta de documentación para aplicar en la construcción de tarea de investigación.}
Escogeremos LaTeX debido a que es una herramienta la cual es muy personalizable a nivel de documentación y porque es
para documentos de texto plano, que es en lo que se basa la tarea de investigación. 

\subsection{Instalación de LaTeX}
\textbf{Paso 1: Descargar e instalar LaTeX}

Iremos al sitio oficial de LaTeX \href{https://www.latex-project.org/}{\textstyleInternetlink{LaTeX - A document
preparation system (latex-project.org)}} donde encontraremos la forma oficial de instalar la herramienta
\href{https://www.tug.org/interest.html#free}{\textstyleInternetlink{TeX Resources on the Web - TeX Users Group
(tug.org)}} en el sitio se encuentran diferentes link de descargar para Windows, MacOS y Linux.

\textbf{Paso 2 (Opcional): Instalar un editor de LaTeX.}

Podemos encontrar diferentes editores de LaTeX en la web y utilizarlo de manera online en los navegadores, aunque es
recomendable utilizar un editor de LaTeX dedicado.

\textbf{Paso 3: Iniciar a escribir y compilar un documento LaTeX.}

\begin{itemize}[series=listWWNumxix,]
\item Abre el editor de LaTeX.
\item Crear un documento de LaTeX de prueba o utilizaremos uno existente.
\item Escribe tu primer código de prueba, ejemplo:
\end{itemize}
{\textbackslash}documentclass\{article\} \ \% Define el tipo de documento (artículo)

{\textbackslash}title\{Ejemplo de Documento LaTeX\} \ \% Título del documento

{\textbackslash}author\{Tú Nombre\} \ \% Tu nombre como autor

{\textbackslash}date\{{\textbackslash}today\} \ \% La fecha actual

{\textbackslash}begin\{document\} \ \% Comienza el documento

{\textbackslash}maketitle \ \% Imprime el título, autor y fecha

{\textbackslash}section\{Introducción\} \ \% Comienza una sección

Este es un ejemplo básico de un documento LaTeX. Aquí puedes ver cómo crear una estructura simple y algunos elementos de
formato de texto.

{\textbackslash}section\{Formato de Texto\} \ \% Otra sección

Puedes {\textbackslash}textbf\{resaltar texto en negrita\} o {\textbackslash}textit\{en cursiva\} utilizando comandos
específicos.

\begin{itemize}[series=listWWNumxx,]
\item Guardar el archivo como “.tex”.
\item Compilamos el documento para generar un archivo PDF, este proceso debería tardar unos segundos, pero depende mucho
del editor que se esté utilizando.
\end{itemize}
\textbf{Paso 4: Visualizar documento PDF}

Una vez se haya compilado el documento podremos visualizarlo en forma de PDF, podemos abrir este documento y visualizar
si todo esta como lo hemos colocado y de esta manera confirmar que LaTeX esta funcionando de manera correcta.

\subsubsection{Video de instalación de LaTeX}
\url{https://youtu.be/T81ncFZ_8uQ?si=cRjov9Zutg9noqrr}

\subsection{Justificación de uso de las herramientas.}
\textbf{Selenium WebDrive}

He escogido esta herramienta de pruebas, debido a que tiene un amplio soporte de navegadores, un tema que es de suma
importancia debido a que un sistema desplegado en la web no se utilizara de un solo navegador en específico, además,
podemos automatizar las pruebas que queramos realizar y es compatible con diferentes lenguajes de programación.

La documentación de la misma es muy basta y la comunidad que la rodea es muy activa lo cual es de suma importancia
debido a que los problemas que podemos llegar a presentar pueden ser resueltos con mucha facilidad y que las personas
de la comunidad nos ayuden a resolverlo 

\textbf{LaTeX}

Es una herramienta de texto plano por lo que es significativamente lo más apegado a lo que necesitamos realizar, además
tenemos muchos comandos de etiqueta con lo cual tenemos un gran catálogo de fuentes, además es utilizado para
documentación extensa y permite la inclusión de figuras y diferentes formas. Contiene una comunidad activa la cual
muestra la utilización de diferentes etiquetas y la personalización de estas.


\bibliographystyle{plain}
\bibliography{referencias.bib}
\cite{libro1}
\cite{libro2}
\cite{libro3}
\cite{libro4}
\cite{libro5}

\end{document}